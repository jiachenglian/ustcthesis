% !TeX root = ../main.tex

\chapter{总结}

\section{成果和收获}
本文通过调研三种图神经网络算法:图卷积网络算法、GraphSAGE算法和DiffPool算法,总结出四个图神经网络运算的基本算子:采样、聚集、组合和池化。
在了解一种混合结构的图神经网络加速器之后,针对硬件特点和算子特性进行了指令设计和指令生成的工作,并为生成的指令做了软件流水优化。

在刚开始写论文时我只对传统的神经网络有一些大致的认识,如卷积神经网络、循环神经网络、长短期记忆网络(LSTM)等,图神经网络对我来说是一个全新的概念。
通过一个学期的学习研究,我对图神经网络的算法和硬件都有了较多的了解,对编译和软件流水优化有了更深入的理解,阅读外文文献的速度也有了提高。

\section{不足和改进}
由于在前期调研图神经网络的相关算法和图神经网络加速器的硬件结构时花费了不少时间,加之疫情原因推迟了课题的开始,留给后续指令生成部分的时间比预期要少。
所以在第四章第二节的算法代码生成中只进行了图卷积网络算法的代码生成和软件流水优化,没有对实际性能的提升进行测试。
在后续的改进中,我会为图神经网络加速器编写模拟器代码,并在此基础上对本文生成的指令和经软件流水优化之后的指令进行性能测试。
另外还会考虑对本文设计的指令在其他架构的图神经网络加速器上的表现进行测试,并进一步对指令进行优化和改进。

图神经网络有着非常广阔的研究前景,然而尽管图神经网络已经在很多领域取得了成功,它并不能为任意图数据提供令人满意的解决方案,比如动态图数据。
静态图数据有着稳定的结构,因此对其建模相较于动态图容易很多。当边和顶点出现或消失时,图神经网络难以自适应地作出改变,目前对能处理动态图的图神经网络算法的研究还处于起步阶段。
希望在不久的将来,图神经网络会像卷积神经网络一样大放异彩。