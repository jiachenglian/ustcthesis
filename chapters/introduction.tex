% !TeX root = ../main.tex

\chapter{引言}

\section{研究的背景与意义}

近年来,得益于神经网络的快速发展,很多机器学习问题已经有了较好的解决方法,在图像识别、机器翻译等很多领域取得了突破性的进展。
神经网络在欧几里得结构数据(如图像、视频、语音等)上取得巨大成功之后,人们开始关注非欧几里得结构数据(如知识图谱数据、蛋白质结构数据等)的处理方法。
这时图神经网络便应运而生,并在处理图数据方面相较于传统神经网络展现出了很大的优势,对图神经网络的研究也越来越受到学术界和工业界的关注。

与神经网络处理器NPU类似,为了加速图神经网络的计算,图神经网络加速器开始出现。
但目前还没有学术论文讨论图神经网络加速器相关的指令,在此背景下,本文希望通过介绍图神经网络的指令生成和优化,来作为对图神经网络研究的补充和完善。

\section{主要工作}

本文的主要工作分为以下几个部分
\begin{enumerate}
    \item 调研现有的图神经网络算法,梳理介绍图卷积网络、GraphSAGE算法、DiffPool算法这三个主流算法的计算流程,并提炼出三个重要算法的共性,确定后续工作所需要的四个基本算子:采样、聚集、组合、池化。
    \item 调研一种混合结构图神经网络加速器的体系结构,了解其主要功能部件的作用和具体的加速方法,大致理解图神经网络算法在硬件上的执行过程。
    \item 针对图神经网络加速器架构和之前总结出的算子设计指令,并为四个算子和三个具体算法进行指令生成,再对指令序列进行软件流水优化。
\end{enumerate}

\section{论文结构}

第一章\ 引言。本章介绍了图神经网络算法和图神经网络加速器的研究背景,阐述了本课题的研究意义,概述了本文所做的主要工作。

第二章\ 图神经网络算法。本章介绍了图神经网络的结构和三个主流图神经网络算法:图卷积网络算法、GraphSAGE算法和DiffPool算法。
通过分析三个算法的共性提炼出采样、聚集、组合、池化四个算子,为代码生成的工作奠定基础。
最后简单介绍了图神经网络框架DGL的一些情况。

第三章\ 图神经网络加速器。本章介绍了一种混合结构的图神经网络加速器,主要讲述了其整体架构、两个关键过程聚集和组合的硬件级优化和一些全局的优化。

第四章\ 代码生成。本章介绍了针对第二章提炼出的算子和第三章的加速器设计的指令,为算子和图卷积网络算法进行了指令生成的工作,并对生成的指令进行软件流水优化。

第五章\ 总结。本章介绍了论文的主要贡献和存在的不足之处,提出了改进建议,并对图神经网络的缺陷进行了说明,对未来进行了展望。